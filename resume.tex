% Zach Morgan's Resume
% Adapted from Andrew McNabb's Resume


\documentclass[11pt,oneside]{article}
\usepackage{geometry}
\usepackage[T1]{fontenc}

\pagestyle{empty}
\geometry{letterpaper,tmargin=1in,bmargin=1in,lmargin=1in,rmargin=1in,headheight=0in,headsep=0in,footskip=.3in}

\setlength{\parindent}{0in}
\setlength{\parskip}{0in}
\setlength{\itemsep}{0in}
\setlength{\topsep}{0in}
\setlength{\tabcolsep}{0in}

% Name and contact information
\newcommand{\name}{Zach Morgan}
\newcommand{\addr}{3724-E Groometown Rd, Greensboro, NC 27407}
\newcommand{\phone}{(919) 265-7487}
\newcommand{\email}{zpmorgan@gmail.com}


%%%%%%%%%%%%%%%%%%%%%%%%%%%%%%%%%%%%%%%%%%%%%%%%%%%%%%%%%
% New commands and environments

% This defines how the name looks
\newcommand{\bigname}[1]{
	\begin{center}\fontfamily{phv}\selectfont\Huge\scshape#1\end{center}
}

% A ressection is a main section (<H1>Section</H1>)
\newenvironment{ressection}[1]{
	\vspace{2pt}
	{\fontfamily{phv}\selectfont\Large#1}
	\begin{itemize}
	\vspace{3pt}
}{
	\end{itemize}
}

% A resitem is a simple list element in a ressection (first level)
\newcommand{\resitem}[1]{
	\vspace{-4pt}
	\item \begin{flushleft} #1 \end{flushleft}
}

% A ressubitem is a simple list element in anything but a ressection (second level)
\newcommand{\ressubitem}[1]{
	\vspace{-1pt}
	\item \begin{flushleft} #1 \end{flushleft}
}

% A resbigitem is a complex list element for stuff like jobs and education:
%  Arg 1: Name of company or university
%  Arg 2: Location
%  Arg 3: Title and/or date range
\newcommand{\resbigitem}[3]{
	\vspace{-5pt}
	\item
	\textbf{#1}---#2 \\
	\textit{#3}
}

% This is a list that comes with a resbigitem
\newenvironment{ressubsec}[3]{
	\resbigitem{#1}{#2}{#3}
	\vspace{-2pt}
	\begin{itemize}
}{
	\end{itemize}
}

% This is a simple sublist
\newenvironment{reslist}[1]{
	\resitem{\textbf{#1}}
	\vspace{-5pt}
	\begin{itemize}
}{
	\end{itemize}
}



%%%%%%%%%%%%%%%%%%%%%%%%%%%%%%%%%%%%%%%%%%%%%%%%%%%%%%%%%
% Now for the actual document:

\begin{document}

\fontfamily{ppl} \selectfont

% Name with horizontal rule
\bigname{\name}

\vspace{-8pt} \rule{\textwidth}{1pt}

\vspace{-1pt} {\small\itshape \addr \hfill \phone; \email}

\vspace{8 pt}




%%%%%%%%%%%%%%%%%%%%%%%%
\begin{ressection}{Objective}

   \resitem{I am seeking open-ended opportunities with potential to further my skillset \& experience in unanticipated directions.}
	
\end{ressection}


%%%%%%%%%%%%%%%%%%%%%%%%
\begin{ressection}{Education}

	\begin{ressubsec}{B.S. in Computer Science, December 2009}{The University of North Carolina at Greensboro}{}
		%\ressubitem{GPA: 3.77 (overall), 3.77 (CS major), 3.90 (Math major)}
		%\ressubitem{Graduation Date: December, 2009}
	\end{ressubsec}

\end{ressection}


%%%%%%%%%%%%%%%%%%%%%%%%
\begin{ressection}{Skills}

	\resitem{\textbf{Operating Systems:} Linux (Ubuntu, Debian), Windows 2000/XP}

	\begin{reslist}{Computer Languages:}

		\ressubitem{Proficient in Perl, Regular expressions, SQL, C, C++, HTML}

		\ressubitem{Familiar with Javascript, \LaTeX, Java, Bash}

	\end{reslist}

	\begin{reslist}{Tools and Systems:}

		\ressubitem{Proficient in Catalyst, DBIx::Class, Template::Toolkit, Moose, SQLite, Git, Bazaar, GTK+}

      \ressubitem{Familiar with Vim, MySQL, Lighttpd, Android, Machine learning concepts, PDL}

	\end{reslist}


\end{ressection}


%%%%%%%%%%%%%%%%%%%%%%%%
\begin{ressection}{Achievements and Activities}

	\resitem{Officer of the UNCG Chapter of Association for Computing Machinery}

	\resitem{Earned Life Scout Rank, Boy Scouts of America (2003)}


\end{ressection}


%%%%%%%%%%%%%%%%%%%%%%%%
\begin{ressection}{Projects}

	\begin{ressubsec}{Basilisk Go Server}{Senior Project at UNCG was to build a a successful correspondence
         board game server in Perl. It demonstrates proficiency with object-relational mapping (ORM),
         web templating with forms and javascript, version control, regular expressions.}{}
	%%%	\ressubitem{Server is at http://www.basiliskgo.com}
	%%%	\ressubitem{Source with revision history is at http://github.com/zpmorgan/basilisk}
	\end{ressubsec}
	\begin{ressubsec}{Collision::2D}{Collision::2D is a float-precision continuous collision detection
      system for some geometric shapes in 2D; it detects collisions between
      points, circles, and rects of any size and velocity.
      Originally implemented on Moose, it was ported to an XS backend.}{}
	\end{ressubsec}
	\begin{ressubsec}{AI::Nerl}{
      AI::Nerl is a perceptron-type neural network library. AI::Nerl uses PDL, the Perl Data
      Language for fast linear operations, including training using backpropagation.
      AI::Nerl has been used to build a digit classifier and a general image classifier.
      }{}
	\end{ressubsec}

\end{ressection}


\end{document}
